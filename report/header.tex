% !TEX encoding = UTF-8 Unicode
% !TEX root = report.tex
% 
\documentclass[a4paper,11pt,twoside,titlepage,abstract,numbers=noenddot,automark,mnsy,intlimits,rgb,dvipsnames]{scrartcl}
%\usepackage[utf8]{inputenc}
\usepackage{csquotes}
\usepackage[\idioma, es-tabla]{babel}
\spanishdecimal{.}
\usepackage{fontspec}
\defaultfontfeatures{Scale=MatchLowercase, Ligatures=TeX}
\usepackage{pdfpages}
\usepackage{fancyvrb}
\usepackage{amssymb}
\usepackage{amsmath}
\usepackage{mathtools}
\usepackage{unicode-math}
\unimathsetup{math-style=ISO,vargreek-shape=unicode}
\usepackage{xunicode}
\usepackage{ifxetex}
\usepackage{algorithm}
\usepackage{algpseudocode}

\ifxetex
  \usepackage{xltxtra}
\fi
\usepackage{verbatim}

\usepackage[binary-units]{siunitx}
\sisetup{
  product-units=single,
  list-units=single,
  per-mode=symbol,
  list-final-separator = { y },
  list-pair-separator = { y },
  range-phrase = { a },
}

\usepackage{fullpage}
\usepackage{framed}
\usepackage{xfrac}

\defaultfontfeatures{Scale=MatchLowercase}
\setmainfont[Ligatures=TeX,
BoldFont=texgyrepagella-bold.otf,
BoldItalicFont=texgyrepagella-bolditalic.otf,
ItalicFont=texgyrepagella-italic.otf]{texgyrepagella-regular.otf}
\setsansfont[Ligatures=TeX,
BoldFont=lmsans10-bold.otf,
BoldItalicFont=lmsans10-boldoblique.otf,
ItalicFont=lmsans10-oblique.otf]{lmsans10-regular.otf}
\setmonofont[BoldFont=lmmonolt10-bold.otf,
BoldItalicFont=lmmonolt10-boldoblique.otf,
ItalicFont=lmmono10-italic.otf,
SlantedFont=lmmonoslant10-regular.otf]{lmmono10-regular.otf}
\setmathfont{texgyrepagella-math.otf}
\setmathfont[range={\mathcal,\mathbfcal},StylisticSet=1]{xits-math.otf}

\usepackage[super]{nth}
%\setmathfont[ Path=fonts/, ]{LM Math}
%\usepackage{natbib}
\usepackage[natbib=true,language=english,style=numeric,citestyle=numeric,bibstyle=numeric,hyperref=true]{biblatex}

\usepackage{url}
\usepackage{pdflscape}
\usepackage{enumitem}

\usepackage{graphicx}
\usepackage{float}
\usepackage{caption}
\usepackage{subcaption}
\usepackage{multicol}
\usepackage{booktabs}

\usepackage[hidelinks]{hyperref}
\hypersetup{
    pdfencoding=auto,
    pdffitwindow=false,      % page fit to window when opened
    pdftitle={\materia\ :: \titol},    % title
    pdfauthor={\pdfauthors},     % author
    pdfsubject={},   % subject of the document
    pdfcreator={XeLaTeX + Hyperref package},
    colorlinks=false
}

\usepackage{setspace}

\usepackage[nouppercase]{scrpage2}

\setlength{\headheight}{15pt}
\renewcommand{\headfont}{\upshape}
\defpagestyle{curr}
  {    %% superior
    (\textwidth,0pt) %líneas
    {    %%par
      {\autors{l}}
      {\hfill}
      {\leftmark}
    }
    {	%%impar
      {\rightmark}
      {\hfill}
      {\autors{r}}
    }
    {	%% una sola cara
      {\thepart}
      {\hfill}
      {\autors{r}}
    }
    (\textwidth,0.5pt) %líneas
  }
  {		%% inferior
    (\textwidth,0.5pt) %líneas
    {	%%par
      {\thepage}
      {\hfill}
      {\materia}
    }
    {	%%impar
      {\titol}
      {\hfill}
      {\thepage}
    }
    {	%% una sola cara
      {\materia: \titol}
      {\hfill}
      {\thepage}
    }
    (\textwidth,0pt) %líneas
  }

\pagestyle{curr}

\headsep = 15pt
%\addtolength{\footskip}{-16pt}
%\addtolength{\textheight}{+16pt}
\addtolength{\topmargin}{-15pt}
\addtolength{\hoffset}{12mm}
\addtolength{\textwidth}{-11mm}
\usepackage{wrapfig}

\usepackage{amsthm}
%~ \theoremprework {\textcolor{white} {\rule{0.2in}{0.11in}} \hrule\rule{0.2in}{0.11in}}
%~ \theorempostwork {\hrule %\textcolor{white} {\rule{0.2in}{0.11in}}
%~ }

\makeatletter
\newcommand{\strong}[1]{\@strong{#1}}
\newcommand{\@@strong}[1]{\textbf{\let\@strong\@@@strong#1}}
\newcommand{\@@@strong}[1]{\textnormal{\let\@strong\@@strong#1}}
\let\@strong\@@strong
\makeatother

%~ \theoremindent0.5cm
%~ \theoremstyle{break}
%~ \theorembodyfont{}
%~ \newtheorem*{defi}{}

\theoremstyle{plain}% default
\newtheorem{thm}{Teorema}[section]
\newtheorem{lem}[thm]{Lemma}
\newtheorem{prop}{Proposition}
\newtheorem{cor}{Corollary}

\theoremstyle{definition}
\newtheorem{defn}{Definition}[section]
\newtheorem{conj}{Conjecture}[section]
\newtheorem{exmp}{Example}[section]

\theoremstyle{remark}
\newtheorem*{obs}{Observation}
\newtheorem*{note}{Remark}
\newtheorem{case}{Case}
\newtheorem*{notation}{Notation}

\addtolength{\voffset}{-15pt}
\addtolength{\headsep}{10pt}
\addtolength{\textheight}{35pt}
\addtolength{\footskip}{-20pt}
\addtolength{\textwidth}{15pt}
\addtolength{\marginparwidth}{-20pt}
\addtolength{\oddsidemargin}{-20pt}
\addtolength{\evensidemargin}{-20pt}

\usepackage{listings}

\definecolor{FonsCodi}{cmyk}{0,0,0,0.04}
\definecolor{Comentaris}{cmyk}{0,0,0,0.6}
\definecolor{mygreen}{rgb}{0,0.6,0}
\definecolor{mygray}{rgb}{0.6,0.6,0.6}
\definecolor{mymauve}{rgb}{0.58,0,0.82}
\definecolor{darkgreen}{rgb}{0.2,0.5,0.2}
\definecolor{orange}{rgb}{1,0.5,0}

\lstset{ %
basicstyle=\ttfamily\small,
numbers=none,                   % where to put the line-numbers
backgroundcolor=\color{FonsCodi},  % choose the background color. You must add \usepackage{color}
rulesepcolor=\color{FonsCodi},
lineskip=-2.5pt,
showspaces=false,               % show spaces adding particular underscores
showstringspaces=false,         % underline spaces within strings
showtabs=false,                 % show tabs within strings adding particular underscores
frame=single,                    % adds a frame around the code
tabsize=8,	                % sets default tabsize to 2 spaces
captionpos=t,                   % sets the caption-position to top
breaklines=true,                % sets automatic line breaking
breakatwhitespace=true,        % sets if automatic breaks should only happen at whitespace
escapeinside={\%*}{*)},          % if you want to add a comment within your code
columns=flexible
}

\renewcommand{\lstlistlistingname}{Lista de ejecuciones}
\renewcommand{\lstlistingname}{Ejecución}
