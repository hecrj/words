% !TEX program = lualatex
% !TEX encoding = UTF-8 Unicode
% 
\newcommand{\titol}{Estimación de cardinalidad}
\newcommand{\materia}{Algorítmica}
\newcommand{\idioma}{english,spanish}
\newcommand{\pdfauthors}{Héctor Ramón Jiménez, Xavier Serra Alza}
\newcommand{\autors}[1]{\begin{tabular}{#1} Ramón -- Serra\end{tabular}}
\newcommand{\data}{\today}
% !TEX encoding = UTF-8 Unicode
% !TEX root = report.tex
% 
\documentclass[a4paper,11pt,twoside,titlepage,abstract,numbers=noenddot,automark,mnsy,intlimits,rgb,dvipsnames]{scrartcl}
%\usepackage[utf8]{inputenc}
\usepackage{csquotes}
\usepackage[\idioma, es-tabla]{babel}
\spanishdecimal{.}
\usepackage{fontspec}
\defaultfontfeatures{Scale=MatchLowercase, Ligatures=TeX}
\usepackage{pdfpages}
\usepackage{fancyvrb}
\usepackage{amssymb}
\usepackage{amsmath}
\usepackage{mathtools}
\usepackage{unicode-math}
\unimathsetup{math-style=ISO,vargreek-shape=unicode}
\usepackage{xunicode}
\usepackage{ifxetex}
\usepackage{algorithm}
\usepackage{algpseudocode}

\ifxetex
  \usepackage{xltxtra}
\fi
\usepackage{verbatim}

\usepackage[binary-units]{siunitx}
\sisetup{
  product-units=single,
  list-units=single,
  per-mode=symbol,
  list-final-separator = { y },
  list-pair-separator = { y },
  range-phrase = { a },
}

\usepackage{fullpage}
\usepackage{framed}
\usepackage{xfrac}

\defaultfontfeatures{Scale=MatchLowercase}
\setmainfont[Ligatures=TeX,
BoldFont=texgyrepagella-bold.otf,
BoldItalicFont=texgyrepagella-bolditalic.otf,
ItalicFont=texgyrepagella-italic.otf]{texgyrepagella-regular.otf}
\setsansfont[Ligatures=TeX,
BoldFont=lmsans10-bold.otf,
BoldItalicFont=lmsans10-boldoblique.otf,
ItalicFont=lmsans10-oblique.otf]{lmsans10-regular.otf}
\setmonofont[BoldFont=lmmonolt10-bold.otf,
BoldItalicFont=lmmonolt10-boldoblique.otf,
ItalicFont=lmmono10-italic.otf,
SlantedFont=lmmonoslant10-regular.otf]{lmmono10-regular.otf}
\setmathfont{texgyrepagella-math.otf}
\setmathfont[range={\mathcal,\mathbfcal},StylisticSet=1]{xits-math.otf}

\usepackage[super]{nth}
%\setmathfont[ Path=fonts/, ]{LM Math}
%\usepackage{natbib}
\usepackage[natbib=true,language=english,style=numeric,citestyle=numeric,bibstyle=numeric,hyperref=true]{biblatex}

\usepackage{url}
\usepackage{pdflscape}
\usepackage{enumitem}

\usepackage{graphicx}
\usepackage{float}
\usepackage{caption}
\usepackage{subcaption}
\usepackage{multicol}
\usepackage{booktabs}

\usepackage[hidelinks]{hyperref}
\hypersetup{
    pdfencoding=auto,
    pdffitwindow=false,      % page fit to window when opened
    pdftitle={\materia\ :: \titol},    % title
    pdfauthor={\pdfauthors},     % author
    pdfsubject={},   % subject of the document
    pdfcreator={XeLaTeX + Hyperref package},
    colorlinks=false
}

\usepackage{setspace}

\usepackage[nouppercase]{scrpage2}

\setlength{\headheight}{15pt}
\renewcommand{\headfont}{\upshape}
\defpagestyle{curr}
  {    %% superior
    (\textwidth,0pt) %líneas
    {    %%par
      {\autors{l}}
      {\hfill}
      {\leftmark}
    }
    {	%%impar
      {\rightmark}
      {\hfill}
      {\autors{r}}
    }
    {	%% una sola cara
      {\thepart}
      {\hfill}
      {\autors{r}}
    }
    (\textwidth,0.5pt) %líneas
  }
  {		%% inferior
    (\textwidth,0.5pt) %líneas
    {	%%par
      {\thepage}
      {\hfill}
      {\materia}
    }
    {	%%impar
      {\titol}
      {\hfill}
      {\thepage}
    }
    {	%% una sola cara
      {\materia: \titol}
      {\hfill}
      {\thepage}
    }
    (\textwidth,0pt) %líneas
  }

\pagestyle{curr}

\headsep = 15pt
%\addtolength{\footskip}{-16pt}
%\addtolength{\textheight}{+16pt}
\addtolength{\topmargin}{-15pt}
\addtolength{\hoffset}{12mm}
\addtolength{\textwidth}{-11mm}
\usepackage{wrapfig}

\usepackage{amsthm}
%~ \theoremprework {\textcolor{white} {\rule{0.2in}{0.11in}} \hrule\rule{0.2in}{0.11in}}
%~ \theorempostwork {\hrule %\textcolor{white} {\rule{0.2in}{0.11in}}
%~ }

\makeatletter
\newcommand{\strong}[1]{\@strong{#1}}
\newcommand{\@@strong}[1]{\textbf{\let\@strong\@@@strong#1}}
\newcommand{\@@@strong}[1]{\textnormal{\let\@strong\@@strong#1}}
\let\@strong\@@strong
\makeatother

%~ \theoremindent0.5cm
%~ \theoremstyle{break}
%~ \theorembodyfont{}
%~ \newtheorem*{defi}{}

\theoremstyle{plain}% default
\newtheorem{thm}{Teorema}[section]
\newtheorem{lem}[thm]{Lemma}
\newtheorem{prop}{Proposition}
\newtheorem{cor}{Corollary}

\theoremstyle{definition}
\newtheorem{defn}{Definition}[section]
\newtheorem{conj}{Conjecture}[section]
\newtheorem{exmp}{Example}[section]

\theoremstyle{remark}
\newtheorem*{obs}{Observation}
\newtheorem*{note}{Remark}
\newtheorem{case}{Case}
\newtheorem*{notation}{Notation}

\addtolength{\voffset}{-15pt}
\addtolength{\headsep}{10pt}
\addtolength{\textheight}{35pt}
\addtolength{\footskip}{-20pt}
\addtolength{\textwidth}{15pt}
\addtolength{\marginparwidth}{-20pt}
\addtolength{\oddsidemargin}{-20pt}
\addtolength{\evensidemargin}{-20pt}

\usepackage{listings}

\definecolor{FonsCodi}{cmyk}{0,0,0,0.04}
\definecolor{Comentaris}{cmyk}{0,0,0,0.6}
\definecolor{mygreen}{rgb}{0,0.6,0}
\definecolor{mygray}{rgb}{0.6,0.6,0.6}
\definecolor{mymauve}{rgb}{0.58,0,0.82}
\definecolor{darkgreen}{rgb}{0.2,0.5,0.2}
\definecolor{orange}{rgb}{1,0.5,0}

\lstset{ %
basicstyle=\ttfamily\small,
numbers=none,                   % where to put the line-numbers
backgroundcolor=\color{FonsCodi},  % choose the background color. You must add \usepackage{color}
rulesepcolor=\color{FonsCodi},
lineskip=-2.5pt,
showspaces=false,               % show spaces adding particular underscores
showstringspaces=false,         % underline spaces within strings
showtabs=false,                 % show tabs within strings adding particular underscores
frame=single,                    % adds a frame around the code
tabsize=8,	                % sets default tabsize to 2 spaces
captionpos=t,                   % sets the caption-position to top
breaklines=true,                % sets automatic line breaking
breakatwhitespace=true,        % sets if automatic breaks should only happen at whitespace
escapeinside={\%*}{*)},          % if you want to add a comment within your code
columns=flexible
}

\renewcommand{\lstlistlistingname}{Lista de ejecuciones}
\renewcommand{\lstlistingname}{Ejecución}

\setstretch{1.0}
\DefineBibliographyStrings{spanish}{%
  references = {Referencias},
}

\title{\materia\\
\Large{\titol}}
\subtitle{Facultat d'Informàtica de Barcelona\\ % Pongo la I mayúscula porque la FIB lo hace
Universitat Politècnica de Catalunya}
\author{
  Héctor Ramón Jiménez \\
  Xavier Serra Alza}
\date{
  \today \\
  cuatrimestre de otoño \\
  curso 2013--2014}

\everymath{\displaystyle}

\newcommand{\CC}{\mathbb{C}}
\newcommand{\RR}{\mathbb{R}}
\newcommand{\NN}{\mathbb{N}}
\newcommand{\bigO}[1]{\ensuremath{\operatorname{O}\left(#1\right)}}% big-O notation/symbol
\newcommand{\bigOmega}[1]{\ensuremath{\operatorname{\Omega}\left(#1\right)}}% big-O notation/symbol
\newcommand{\bigTheta}[1]{\ensuremath{\operatorname{\Theta}\left(#1\right)}}% big-O notation/symbol
\newcommand{\slot}[1]{\textsl{\texttt{#1}}}
\newcommand{\clase}[1]{\texttt{#1}}
\newcommand{\regla}[1]{\textsl{\textsf{#1}}}

\newenvironment{slotlist}{%
   \renewcommand\descriptionlabel[1]{\hspace{\labelsep}\slot{##1}}
   \begin{description}%
}{%
   \end{description}%
}

\bibliography{references}
\begin{document}

\maketitle
\tableofcontents
\listoftables
\listoffigures
\vfill
\cleardoublepage

% !TEX encoding = UTF-8 Unicode
% !TEX root = ../report.tex
% 
\section{Introduccion}
[...]

% !TEX encoding = UTF-8 Unicode
% !TEX root = ../report.tex
% 
\section{Investigación}

La información referente al algoritmo \texttt{HyperLogLog} se ha obtenido principalmente
del artículo \citetitle{hll:HyperLogLog} ~\cite{hll:HyperLogLog}.
El artículo presenta una descripción detallada del algoritmo y sus ventajas en relación a otros
algoritmos de estimación de cardinalidad.

En esta sección se introducen y se explican, de manera general, las bases del algoritmo \texttt{HyperLogLog}.

\subsection{Funciones de hash}
\label{hash}

Una \textbf{función de hash} se define como una función que mapea todo elemento de un
conjunto de tamaño arbitrario $A$ con otro elemento de un conjunto de tamaño finito $B$.
$$f: A \rightarrow B$$

Una \textbf{buena} función de hash debe ser lo más \textbf{uniforme} posible, es decir, debe mapear los elementos de
$A$ tan \textbf{equitativamente} como sea posible sobre $B$. En otras palabras, todo elemento en $B$ debe ser generado
por la función de hash con \textbf{aproximadamente la misma probabilidad}.

Por lo tanto, si $h$ es una buena función de hash:
$$h: \Sigma^* \rightarrow \{0,1\}^r$$

Donde $\Sigma^*$ representa el conjunto de todas las palabras y $\{0, 1\}^r$ es el conjunto de cadenas binarias de $r$ bits.
Entonces, dada una palabra cualquiera $w \in \Sigma^*$, se puede esperar que:

\begin{align*}
  \left.\begin{array}{r@{\mskip\thickmuskip}l}
    p, a \in \{0, 1\} \\
    p \neq a \\
    1 \leq k \leq r
  \end{array} \right\}
  \quad \implies \quad
  \left.
    P\left( h(w) = p^{k-1} \: a \: b_{k+1} \: b_{k+2} \; ... \; b_r \right) = 2^{-k}
  \right.
\end{align*}

Esto significa que la probabilidad de que la \textbf{primera ocurrencia} de $1$ o $0$ en $h(w)$ sea en el bit $k$ es $2^{-k}$.
Es decir, hay un 50\% de probabilidades de que $h(w)$ empiece por 1, un 25\% de que empiece por $01$, un 12.5\% de que
empiece por $001$, y así sucesivamente.

\subsection{Descripción general del algoritmo}
\label{investigacion:descripcion}

La idea principal de \texttt{HyperLogLog} gira en torno a la propiedad de \textbf{uniformidad} que presentan las buenas
funciones de hash.

Dada una buena función de hash $h$ (igual que en la sección \ref{hash}):
$$h: \Sigma^* \rightarrow \{0,1\}^r$$

Y si el multiconjunto de entrada $M$ tiene $N$ elementos y $n$ elementos únicos, entonces puede esperarse que:

\begin{align*}
  \left.\begin{array}{r@{\mskip\thickmuskip}l}
    p, a \in \{0, 1\} \\
    p \neq a \\
    N \geq 2 \\
    k = \floor{log_2(N)}
  \end{array} \right\}
  \quad \implies \quad
  \left.
    \exists{w \in M}\!: h(w) = p^{k-1} \: a \: b_{k+1} \: b_{k+2} \: ... \: b_r
  \right.
\end{align*}

Por ejemplo, si $M$ tiene \textbf{4 elementos} ($N = 4$) entonces se puede esperar que \textbf{exista uno tal que su hash
empiece por 01}. Si $N = 8$ entonces un elemento debería empezar con la secuencia 001, y así sucesivamente.

Lo interesante, sin embargo, es \textbf{hacerlo a la inversa}.
Sea $a \in \{0, 1\}$ y $k_{max}$ la posición \textbf{más tardía} en la que se ha observado
una ocurrencia de $a$ para todo hash de todo elemento de $M$ entonces se puede estimar que la cardinalidad $n$ de $M$ sea $E = 2^{k_{max}}$.
Por ejemplo, si tras aplicar $h$ a todo elemento de $M$ se observa que la posición más tardía en la que se ha observado un
$1$ es 3 (el elemento empieza con la secuencia $\textbf{001}$), entonces se puede estimar que $n$ es $E = 2^3 = \textbf{8}$ \textbf{elementos}.

Sin embargo, esta estimación es realmente \textbf{imprecisa}. No es difícil ver que con este estimador
\textbf{la cardinalidad se reduce a simples potencias de 2}. Para eliminar esta imprecisión se usa una tabla $T$ de $m$ entradas
para guardar diferentes $k_{max}$, obteniendo el índice $i$ a partir de los primeros $\floor{log_2(m)}$ bits del hash devuelto
y $k_{candidato}$ con los restantes. A partir de $T$ es posible mejorar el estimador anterior:

$$ E = 2 ^ { \frac{1}{m} \sum\limits_{i=0}^{m}\! T[i] } \cdot m $$

Sin embargo, un análisis estadístico realizado por \emph{Flajolet} ~\cite{hll:HyperLogLog} muestra que este estimador
tiene cierta \textbf{tendencia a hacer estimaciones grandes}. Para corregirlo, la estimación se suele multiplicar por la
\textbf{constante 0.79402}, encontrada de manera \textbf{experimental} por el mismo \emph{Flajolet}.
Este estimador es el que utiliza el algoritmo \texttt{LogLog}, una versión anterior a \texttt{HyperLogLog}, el cual tiene
\textbf{un error estándar} de  $\frac{1.30}{\sqrt{m}}$.

\texttt{HyperLogLog} se diferencia del algoritmo \texttt{LogLog} en que aplica la \textbf{media armónica} en vez de la
media aritmética, obteniendo la misma precisión usando solo el $64\%$ de \textbf{memoria que LogLog}, y \textbf{reduciendo así el error estándar} hasta $\frac{1.04}{\sqrt{m}}$.

Por último, si la estimación resultante es muy alta o muy baja entonces se lleva a cabo una corrección:

\begin{description}
\item[Estimación baja] Si $E < \frac{5m}{2}$, se pueden haber dado casos en que haya \textbf{posiciones vacías en la tabla} que
influyan en el valor de la estimación. En éste caso, se cuentan cuantas de estas posiciones vacías
hay, y en caso de que haya por lo menos una, se usa un nuevo valor $E*$ para la estimación: 
$$E* = m \cdot log\left(\frac{m}{V}\right)$$

Siendo $V$ el número de posiciones vacías. Esta fórmula viene dada por las propiedades de las
\textbf{asignaciones aleatorias}. Éstas indican que si $n$ pelotas son lanzadas aleatoriamente a $m$ canastas entonces
se puede esperar que el número de canastas vacías sea $m \cdot e^{-\mu}$, donde $\mu = n / m$. Por tanto, si se observan
$V$ posiciones vacías sobre un total de $m$ es de esperar que $\mu$ sea cercano a $\log(m/V)$, por lo que $n$ estará
cerca de $m \cdot log(m/v)$.

\item[Estimación alta] Asumiendo una función de hash de 32 bits. Si $E > \frac{2^{32}}{30}$, es de esperar que se hayan
producido \textbf{muchas colisiones} que hayan afectado a la estimación final. La corrección consiste en usar la siguiente $E*$
como sustituto de $E$:
$$E* = -2^{32} \cdot log\left(1 - \frac{E}{2^{32}}\right)$$

Para esta fórmula, se usa el mismo modelo de las canastas del punto anterior, pero se sustituye $m$
por $2^r$, siendo $r$ el número de bits usados en la función de hash. Es decir, $E$ \textbf{estima el número de valores diferentes
a los que se les ha aplicado la función de hash}, que será, con una alta probabilidad, próximo a $2^r \cdot (1 - e^{\frac{-n}{2^r}})$.
Y aislando: $n=-2^r \cdot log\left(1 - \frac{E}{2^r}\right)$.
\end{description}

El algoritmo \ref{algoritmo:hyper1} muestra la versión de \texttt{HyperLogLog} descrita en esta sección.

\begin{algorithm}[h!]
\caption{\texttt{HyperLogLog} para funciones de hash de 32 bits}
\label{algoritmo:hyper1}
\textit{Let $h: D\rightarrow \{0,1\}^{32}$ hash data from D to binary 32-bit word.}

\textit{Let $\rho(s)$ be the position of the leftmost 1-bit of s: e.g.,
$\rho(1...) = 1, \rho(0001...) = 4, \rho(0^K) = K + 1$.}

\textbf{define} $\alpha_{16}=0.673;\alpha_{32}=0.697;\alpha_{64}=0.709;\alpha_m=0.7213/(1+1.079/m)$
for $m \geq 128;$

\textbf{Program \texttt{HYPERLOGLOG}} (\textbf{input} $M$: multiset of items from domain $D$).

\textbf{assume} $m=2^b$ with $ b\in[4..16]$.

\textbf{initialize} a collection of $m$ registers, $M[1],...,M[m]$, to 0;

\begin{algorithmic}
    \FOR{$v\in M$}
            \STATE $x  := h(v)$
            \STATE $j   := 1 + (x_1 x_2 ... x_b)_2$ \COMMENT{binary address determined by the first b bits of x}
            \STATE $w := x_{b+1} x_{b+2} ... $
            \STATE $M[j] := max(M[j],\rho(w))$
    \ENDFOR

    \STATE $E:=\alpha _m m^2·\left(\sum\limits_{j=1}^m 2^{-M[j]}\right)^{-1}$ \COMMENT{the raw HyperLogLog estimate}
    \IF{$E \leq \frac{5}{2}m$}
        \STATE $V :=$ the number of registers equal to $0$
        \STATE \algorithmicif\ $V \neq 0$ \algorithmicthen\ $E* := m \cdot log(m / V)$ \algorithmicelse\ $E* := E$
        \COMMENT{small range correction}
    \ENDIF
    \IF{$E\leq \frac{1}{30}2^{32}$}
        \STATE $E*:=E$ \COMMENT{intermediate range -- no correction}
    \ELSE
        \STATE $E* := -2^{32}log(1-E/2^{32})$ \COMMENT{large range correction}
    \ENDIF
    \RETURN{cardinality estimate E* with typical relative error $\pm$ 1.04/$\sqrt{m}$}
\end{algorithmic}
\end{algorithm}

% !TEX encoding = UTF-8 Unicode
% !TEX root = ../report.tex
% 
\section{Análisis estadístico del algoritmo}
\label{analisis}

\subsection{Memoria fijada a 1024 bytes}

En esta sección se estudia el comportamiento del algoritmo \texttt{HyperLogLog} en \textbf{9 \emph{datasets}} distintos limitando
la cantidad de memoria utilizada a \textbf{1024 bytes}. En el apéndice \ref{graficas} se incluyen gráficas que muestran los resultados
obtenidos para cada \emph{dataset} de forma detallada. Lo importante, sin embargo, es
\textbf{observar los resultados obtenidos de manera general para cada muestra}.

Se presentan las tablas \ref{tabla:resumen_1024} y \ref{tabla:count_1024} a modo de resumen para cada \emph{dataset}.
\clearpage

\begin{table}[h!]
    \centering
    \begin{tabular}{l r r r S S S}
    \strong{Dataset} & \strong{n} & \strong{N} & \strong{Est. media} &
    \strong{SE} & \textbf{T. medio ($ms$)} & \textbf{T. elem. ($\mu s$)}\\ \hline
    \newcounter{dataset}
\forloop{dataset}{1}{\value{dataset} < 10}{
\textbf{D\arabic{dataset}} &
\input{../data/D\arabic{dataset}/summary_1024.tex}
}
\end{tabular}
    \caption{Memoria fijada a 1024 bytes. Resumen de los resultados.}
    \label{tabla:resumen_1024}
\end{table}

\begin{table}[h!]
    \centering
    \begin{tabular}{l r r r r r r}
    \strong{Dataset} & \strong{[0, 1)} & \strong{[1, 5)} & \strong{[5, 10)} &
    \strong{[10, 15)} & \textbf{[15, 20)} & \textbf{[20, 100]} \\ \hline
\forloop{dataset}{1}{\value{dataset} < 10}{
\textbf{D\arabic{dataset}} &
\input{../data/D\arabic{dataset}/count_1024.tex}
}
\end{tabular}
    \caption{Memoria fijada a 1024 bytes. Clasificación de las ejecuciones según el error relativo (\%).}
    \label{tabla:count_1024}
\end{table}

[Aquí comentario sobre el SE cerca del valor esperado y otras cosas interesantes que puedas observar]

\subsection{Influencia de la memoria disponible}

\begin{table}[h!]
    \centering
    \begin{tabular}{l r r r S S S}
    \strong{Memoria} & \strong{n} & \strong{N} & \strong{Est. media} &
    \strong{SE} & \textbf{T. medio ($ms$)} & \textbf{T. elem. ($\mu s$)}\\ \hline

\textbf{32} & 3185 & 17219 & 3209 & 0.194 & 4.849 & 0.282\\ \hline

\textbf{64} & 3185 & 17219 & 3182 & 0.129 & 4.392 & 0.255\\ \hline

\textbf{128} & 3185 & 17219 & 3181 & 0.085 & 4.631 & 0.269\\ \hline

\textbf{256} & 3185 & 17219 & 3190 & 0.077 & 4.341 & 0.252\\ \hline

\textbf{512} & 3185 & 17219 & 3187 & 0.049 & 4.319 & 0.251\\ \hline

\textbf{1024} & 3185 & 17219 & 3210 & 0.027 & 4.850 & 0.282\\ \hline

\textbf{2048} & 3185 & 17219 & 3186 & 0.026 & 4.901 & 0.285\\ \hline

\textbf{4096} & 3185 & 17219 & 3182 & 0.015 & 3.795 & 0.220\\ \hline

\textbf{8192} & 3185 & 17219 & 3186 & 0.024 & 4.420 & 0.257\\ \hline

\textbf{16384} & 3185 & 17219 & 3186 & 0.007 & 4.287 & 0.249\\ \hline


\end{tabular}
    \caption{Influencia de la memoria sobre el dataset D1. Resumen de resultados.}
    \label{tabla:resumen_1024}
\end{table}

\begin{table}[h!]
    \centering
    \begin{tabular}{l r r r r r r}
    \strong{Memoria} & \strong{[0, 1)} & \strong{[1, 5)} & \strong{[5, 10)} &
    \strong{[10, 15)} & \textbf{[15, 20)} & \textbf{[20, 100]} \\ \hline

\textbf{32} & 9 & 27 & 43 & 40 & 25 & 56\\ \hline

\textbf{64} & 9 & 40 & 59 & 46 & 28 & 18\\ \hline

\textbf{128} & 13 & 81 & 60 & 31 & 10 & 5\\ \hline

\textbf{256} & 23 & 74 & 70 & 28 & 2 & 3\\ \hline

\textbf{512} & 26 & 126 & 45 & 3 & 0 & 0\\ \hline

\textbf{1024} & 62 & 122 & 16 & 0 & 0 & 0\\ \hline

\textbf{2048} & 70 & 127 & 3 & 0 & 0 & 0\\ \hline

\textbf{4096} & 106 & 94 & 0 & 0 & 0 & 0\\ \hline

\textbf{8192} & 146 & 54 & 0 & 0 & 0 & 0\\ \hline

\textbf{16384} & 186 & 14 & 0 & 0 & 0 & 0\\ \hline


\end{tabular}
    \caption{Influencia de la memoria sobre el dataset D1. Clasificación de las ejecuciones según el error relativo (\%).}
    \label{tabla:resumen_1024}
\end{table}

% !TEX encoding = UTF-8 Unicode
% !TEX root = ../report.tex
% 
\section{Conclusiones}
\label{conclusiones}

Este documento es el resultado de una investigación realizada sobre un algoritmo de \textbf{estimación de la cardinalidad}:
\texttt{HyperLogLog}. En él se ha descrito de manera general cuáles son las bases de este algoritmo y cómo funciona. Además, se
ha puesto a prueba su precisión bajo un análisis estadístico.

Las \textbf{conclusiones} obtenidas a partir de este ánalisis son las  siguientes:

\begin{itemize}
	\item Las estimaciones realizadas son igualmente \textbf{precisas} independientemente del \emph{dataset}.
	\item \textbf{A más memoria disponible, más precisión}. Sin embargo, mejorar la precisión necesita cada vez de más memoria, hasta un punto en que la mejora puede considerarse inapreciable.
	\item Para obtener estimaciones razonablemente \textbf{precisas}, por debajo del 10\% de error relativo, y en una sola ejecución se
necesitan por lo menos 512 bytes de \textbf{memoria}, pero los mejores resultados se empiezan a obtener a partir de
\textbf{1024 bytes} de \textbf{memoria disponible}.
	\item Con \textbf{memorias reducidas} es posible obtener \textbf{precisiones muy similares} a las obtenidas con
\textbf{memorias mayores} si se lleva a cabo una media entre un \textbf{número suficiente de ejecuciones}, con la desventaja del
incremento en \textbf{tiempo total de ejecución}.
	\item El programa es \textbf{rápido}, siendo capaz de procesar un \emph{dataset} de \textbf{más de 7 millones de elementos en medio segundo}.
\end{itemize}

Finalmente, se concluye que \texttt{HyperLogLog} funciona de manera \textbf{eficaz} y \textbf{eficiente}, obteniendo unos resultados
considerablemente \textbf{precisos} sin necesitar de mucho \textbf{tiempo} ni \textbf{memoria}.


\end{document}
