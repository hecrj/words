% !TEX program = lualatex
% !TEX encoding = UTF-8 Unicode
% 
\newcommand{\titol}{Estimación de la cardinalidad}
\newcommand{\materia}{Algorítmica}
\newcommand{\idioma}{english,spanish}
\newcommand{\pdfauthors}{Héctor Ramón Jiménez, Xavier Serra Alza}
\newcommand{\autors}[1]{\begin{tabular}{#1} Ramón -- Serra\end{tabular}}
\newcommand{\data}{\today}
% !TEX encoding = UTF-8 Unicode
% !TEX root = report.tex
% 
\documentclass[a4paper,11pt,twoside,titlepage,abstract,numbers=noenddot,automark,mnsy,intlimits,rgb,dvipsnames]{scrartcl}
%\usepackage[utf8]{inputenc}
\usepackage{csquotes}
\usepackage[\idioma, es-tabla]{babel}
\spanishdecimal{.}
\usepackage{fontspec}
\defaultfontfeatures{Scale=MatchLowercase, Ligatures=TeX}
\usepackage{pdfpages}
\usepackage{fancyvrb}
\usepackage{amssymb}
\usepackage{amsmath}
\usepackage{mathtools}
\usepackage{unicode-math}
\unimathsetup{math-style=ISO,vargreek-shape=unicode}
\usepackage{xunicode}
\usepackage{ifxetex}
\usepackage{algorithm}
\usepackage{algpseudocode}

\ifxetex
  \usepackage{xltxtra}
\fi
\usepackage{verbatim}

\usepackage[binary-units]{siunitx}
\sisetup{
  product-units=single,
  list-units=single,
  per-mode=symbol,
  list-final-separator = { y },
  list-pair-separator = { y },
  range-phrase = { a },
}

\usepackage{fullpage}
\usepackage{framed}
\usepackage{xfrac}

\defaultfontfeatures{Scale=MatchLowercase}
\setmainfont[Ligatures=TeX,
BoldFont=texgyrepagella-bold.otf,
BoldItalicFont=texgyrepagella-bolditalic.otf,
ItalicFont=texgyrepagella-italic.otf]{texgyrepagella-regular.otf}
\setsansfont[Ligatures=TeX,
BoldFont=lmsans10-bold.otf,
BoldItalicFont=lmsans10-boldoblique.otf,
ItalicFont=lmsans10-oblique.otf]{lmsans10-regular.otf}
\setmonofont[BoldFont=lmmonolt10-bold.otf,
BoldItalicFont=lmmonolt10-boldoblique.otf,
ItalicFont=lmmono10-italic.otf,
SlantedFont=lmmonoslant10-regular.otf]{lmmono10-regular.otf}
\setmathfont{texgyrepagella-math.otf}
\setmathfont[range={\mathcal,\mathbfcal},StylisticSet=1]{xits-math.otf}

\usepackage[super]{nth}
%\setmathfont[ Path=fonts/, ]{LM Math}
%\usepackage{natbib}
\usepackage[natbib=true,language=english,style=numeric,citestyle=numeric,bibstyle=numeric,hyperref=true]{biblatex}

\usepackage{url}
\usepackage{pdflscape}
\usepackage{enumitem}

\usepackage{graphicx}
\usepackage{float}
\usepackage{caption}
\usepackage{subcaption}
\usepackage{multicol}
\usepackage{booktabs}

\usepackage[hidelinks]{hyperref}
\hypersetup{
    pdfencoding=auto,
    pdffitwindow=false,      % page fit to window when opened
    pdftitle={\materia\ :: \titol},    % title
    pdfauthor={\pdfauthors},     % author
    pdfsubject={},   % subject of the document
    pdfcreator={XeLaTeX + Hyperref package},
    colorlinks=false
}

\usepackage{setspace}

\usepackage[nouppercase]{scrpage2}

\setlength{\headheight}{15pt}
\renewcommand{\headfont}{\upshape}
\defpagestyle{curr}
  {    %% superior
    (\textwidth,0pt) %líneas
    {    %%par
      {\autors{l}}
      {\hfill}
      {\leftmark}
    }
    {	%%impar
      {\rightmark}
      {\hfill}
      {\autors{r}}
    }
    {	%% una sola cara
      {\thepart}
      {\hfill}
      {\autors{r}}
    }
    (\textwidth,0.5pt) %líneas
  }
  {		%% inferior
    (\textwidth,0.5pt) %líneas
    {	%%par
      {\thepage}
      {\hfill}
      {\materia}
    }
    {	%%impar
      {\titol}
      {\hfill}
      {\thepage}
    }
    {	%% una sola cara
      {\materia: \titol}
      {\hfill}
      {\thepage}
    }
    (\textwidth,0pt) %líneas
  }

\pagestyle{curr}

\headsep = 15pt
%\addtolength{\footskip}{-16pt}
%\addtolength{\textheight}{+16pt}
\addtolength{\topmargin}{-15pt}
\addtolength{\hoffset}{12mm}
\addtolength{\textwidth}{-11mm}
\usepackage{wrapfig}

\usepackage{amsthm}
%~ \theoremprework {\textcolor{white} {\rule{0.2in}{0.11in}} \hrule\rule{0.2in}{0.11in}}
%~ \theorempostwork {\hrule %\textcolor{white} {\rule{0.2in}{0.11in}}
%~ }

\makeatletter
\newcommand{\strong}[1]{\@strong{#1}}
\newcommand{\@@strong}[1]{\textbf{\let\@strong\@@@strong#1}}
\newcommand{\@@@strong}[1]{\textnormal{\let\@strong\@@strong#1}}
\let\@strong\@@strong
\makeatother

%~ \theoremindent0.5cm
%~ \theoremstyle{break}
%~ \theorembodyfont{}
%~ \newtheorem*{defi}{}

\theoremstyle{plain}% default
\newtheorem{thm}{Teorema}[section]
\newtheorem{lem}[thm]{Lemma}
\newtheorem{prop}{Proposition}
\newtheorem{cor}{Corollary}

\theoremstyle{definition}
\newtheorem{defn}{Definition}[section]
\newtheorem{conj}{Conjecture}[section]
\newtheorem{exmp}{Example}[section]

\theoremstyle{remark}
\newtheorem*{obs}{Observation}
\newtheorem*{note}{Remark}
\newtheorem{case}{Case}
\newtheorem*{notation}{Notation}

\addtolength{\voffset}{-15pt}
\addtolength{\headsep}{10pt}
\addtolength{\textheight}{35pt}
\addtolength{\footskip}{-20pt}
\addtolength{\textwidth}{15pt}
\addtolength{\marginparwidth}{-20pt}
\addtolength{\oddsidemargin}{-20pt}
\addtolength{\evensidemargin}{-20pt}

\usepackage{listings}

\definecolor{FonsCodi}{cmyk}{0,0,0,0.04}
\definecolor{Comentaris}{cmyk}{0,0,0,0.6}
\definecolor{mygreen}{rgb}{0,0.6,0}
\definecolor{mygray}{rgb}{0.6,0.6,0.6}
\definecolor{mymauve}{rgb}{0.58,0,0.82}
\definecolor{darkgreen}{rgb}{0.2,0.5,0.2}
\definecolor{orange}{rgb}{1,0.5,0}

\lstset{ %
basicstyle=\ttfamily\small,
numbers=none,                   % where to put the line-numbers
backgroundcolor=\color{FonsCodi},  % choose the background color. You must add \usepackage{color}
rulesepcolor=\color{FonsCodi},
lineskip=-2.5pt,
showspaces=false,               % show spaces adding particular underscores
showstringspaces=false,         % underline spaces within strings
showtabs=false,                 % show tabs within strings adding particular underscores
frame=single,                    % adds a frame around the code
tabsize=8,	                % sets default tabsize to 2 spaces
captionpos=t,                   % sets the caption-position to top
breaklines=true,                % sets automatic line breaking
breakatwhitespace=true,        % sets if automatic breaks should only happen at whitespace
escapeinside={\%*}{*)},          % if you want to add a comment within your code
columns=flexible
}

\renewcommand{\lstlistlistingname}{Lista de ejecuciones}
\renewcommand{\lstlistingname}{Ejecución}

\setstretch{1.0}
\DefineBibliographyStrings{spanish}{%
  references = {Referencias},
}

\title{\materia\\
\Large{\titol}}
\subtitle{Facultat d'Informàtica de Barcelona\\ % Pongo la I mayúscula porque la FIB lo hace
Universitat Politècnica de Catalunya}
\author{
  Héctor Ramón Jiménez \\
  Xavier Serra Alza}
\date{
  \today \\
  cuatrimestre de otoño \\
  curso 2013--2014}

\everymath{\displaystyle}

\newcommand{\CC}{\mathbb{C}}
\newcommand{\RR}{\mathbb{R}}
\newcommand{\NN}{\mathbb{N}}
\newcommand{\bigO}[1]{\ensuremath{\operatorname{O}\left(#1\right)}}% big-O notation/symbol
\newcommand{\bigOmega}[1]{\ensuremath{\operatorname{\Omega}\left(#1\right)}}% big-O notation/symbol
\newcommand{\bigTheta}[1]{\ensuremath{\operatorname{\Theta}\left(#1\right)}}% big-O notation/symbol
\newcommand{\slot}[1]{\textsl{\texttt{#1}}}
\newcommand{\clase}[1]{\texttt{#1}}
\newcommand{\regla}[1]{\textsl{\textsf{#1}}}

\newenvironment{slotlist}{%
   \renewcommand\descriptionlabel[1]{\hspace{\labelsep}\slot{##1}}
   \begin{description}%
}{%
   \end{description}%
}

\bibliography{references}
\begin{document}

\renewcommand{\listalgorithmname}{Índice de algoritmos}

\maketitle
\tableofcontents
\listofalgorithms
\listoftables
\listoffigures
\vfill
\cleardoublepage

% !TEX encoding = UTF-8 Unicode
% !TEX root = ../report.tex
% 
\section{Introducción}


% !TEX encoding = UTF-8 Unicode
% !TEX root = ../report.tex
% 
\section{Investigación}

La información referente al algoritmo \texttt{HyperLogLog} se ha obtenido principalmente
del artículo \citetitle{hll:HyperLogLog} ~\cite{hll:HyperLogLog}.
El artículo presenta una descripción detallada del algoritmo y sus ventajas en relación a otros
algoritmos de estimación de cardinalidad.

\subsection{Funciones de hash}

La idea principal de \texttt{HyperLogLog} gira en torno a las funciones de hash.
Una buena función de hash debe ser \textbf{uniforme}...

\subsection{Descripción detallada del algoritmo}
aprovecha la unformidad que proporcionan las funciones de hash. La idea básica del algoritmo es la
siguiente:

Entre las propiedades de las funciones de hash está el que los bits de la salida son independientes
y cada uno tiene un 50$\%$ de posibilidades de ocurrir. Teniendo esto en cuenta, se puede esperar
que:

	El $50\%$ del output sea de la forma 1X...X
	
	El $25\%$ del output sea de la forma 01X...X
	
	El $12.5\%$ del output sea de la forma 001X...X
	
	...
	
Por lo que si, por ejemplo, se tienen 8 valores, se puede esperar que 1 sea de la forma 001X...X,
si se tienen 4, 1 será de la forma 01X...X, etc. Y de aquí se salta a la inversa: si la primera
posición (si se empieza a contar desde el primer bit) en la que aparece un 1 és el índice 2 (tomando
el primer bit como índice 0), se puede esperar que haya 8 elementos, si el índice és 3 se puede
esperar que haya 16 elementos y así sucesivamente.

No obstante, éste resultado será, en el mejor de los casos, aproximado, y con un margen de error
muy amplio, por lo que se tiene que optimizar para que sea viable. Para ello se usa una tabla en la
que se guardan varias estimaciones, y en la que se usan los primeros bits del valor de hash para
determinar el índice, y el resto de bits para calcular la estimación. Finalmente, una vez se tienen
todas las estimaciones, se usa, y aquí es donde HyperLogLog se diferencia del algoritmo LogLog,
la media armónica de todas ellas para obtener el valor aproximado final.

\begin{algorithm}[h]
\caption{\texttt{HyperLogLog} para funciones de hash de 32 bits}
\textit{Let $h: D\rightarrow{0,1}^{32}$ hash data from D to binary 32-bit word.}

\textit{Let $\rho(s)$ be the position of the leftmost 1-bit of s: e.g.,
$\rho(1...) = 1, \rho(0001...) = 4, \rho(0^K) = K + 1$.}

\textbf{define} $\alpha_{16}=0.673;\alpha_{32}=0.697;\alpha_{64}=0.709;\alpha_m=0.7213/(1+1.079/m)$
for $m \geq 128;$

\textbf{Program \texttt{HYPERLOGLOG}} (\textbf{input} $M$: multiset of items from domain $D$).

\textbf{assume} $m=2^b$ with $ b\in[4..16]$.

\textbf{initialize} a collection of $m$ registers, $M[1],...,M[m]$, to 0;

\begin{algorithmic}
    \FOR{$v\in M$}
            \STATE $x  := h(v)$
            \STATE $j   := 1 + (x_1 x_2 ... x_b)_2$ \COMMENT{binary address determined by the first b bits of x}
            \STATE $w := x_{b+1} x_{b+2} ... $
            \STATE $M[j] := max(M[j],\rho(w))$
    \ENDFOR

    \STATE $E:=\alpha _m m^2·\left(\sum\limits_{j=1}^m 2^{-M[j]}\right)^{-1}$ \COMMENT{the raw HyperLogLog estimate}
    \IF{$E \leq \frac{5}{2}m$}
        \STATE $V :=$ the number of registers equal to $0$
        \STATE \algorithmicif\ $V \neq 0$ \algorithmicthen\ $E* := m \cdot log(m / V)$ \algorithmicelse\ $E* := E$
        \COMMENT{small range correction}
    \ENDIF
    \IF{$E\leq \frac{1}{30}2^{32}$}
        \STATE $E*:=E$ \COMMENT{intermediate range -- no correction}
    \ELSE
        \STATE $E* := -2^{32}log(1-E/2^{32})$ \COMMENT{large range correction}
    \ENDIF
    \RETURN{cardinality estimate E* with typical relative error $\pm$ 1.04/$\sqrt{m}$}
\end{algorithmic}
\end{algorithm}

Finalmente, como ya se ha comentado, si se tiene una estimación (llamada $E$) con valores muy
grandes o bien muy bajos, se lleva a cabo una corrección. Hay 3 casos:

\begin{enumerate}
\item Si $E < 5m/2$, se pueden haber dado casos en que haya posiciones vacías en la tabla que
perviertan el valor de la estimación. En éste caso, se cuentan cuantas de estas posiciones vacías
hay, y en caso de que haya por lo menos uno, se usa un nuevo valor (E*) para la estimación: 

$$E* = m \cdot log(m/V)$$

Siendo $V$ el número de posiciones vacías. Esta fórmula viene dada por las propiedades de las
asignaciones aleatorias. Éstas indican que, dadas $m$ canastas, y $n$ lanzamientos de pelotas,
se puede esperar que el número de canastas vacías sería $\mu$, con $\mu = n / m$. Por tanto, si
observamos $V$ posiciones vacías sobre un total de $m$
se puede esperar que $\mu$ sea cercano a $\log(m/V)$, por lo que $n$ estará cerca de
$m \cdot log(m/v)$.

\item Si $E > 2^{32}/30$, se producirán demasiadas colisiones en la función de hash que llegarían a
afectar al resultado final. En éste caso se usa la siguiente $E*$ como sustituto de $E$:
$$E* = -2^{32}log(1-E/2^{32})$$
Para esta fórmula, se usa el mismo modelo de las canastas del punto anterior, pero se sustituye $m$
por $2^L$, siendo $L$ el número de bits usados en la función de hash, normalmente 32 para $n
\subset 1...10^9$. Es decir, $E$ estima el número de valores de hash diferente, que será, con una
alta probabilidad, próximo a $2^L(1-e^{-k})$, siendo $k = n/2^L$. Por tanto, si aislamos n nos da
que 
$n=-2^Llog(1-E/2^L)$
\\

\item En el caso de que $E$ esté entre éstos dos valores, la estimación entra dentro de los valores
“normales”, por lo que no hace falta modificarla.

\end{enumerate}

% !TEX encoding = UTF-8 Unicode
% !TEX root = ../report.tex
% 
\section{Análisis estadístico del algoritmo}
\label{analisis}

\subsection{Memoria fijada a 1024 bytes}

En esta sección se estudia el comportamiento del algoritmo \texttt{HyperLogLog} en \textbf{9 \emph{datasets}} distintos limitando
la cantidad de memoria utilizada a \textbf{1024 bytes}. En el apéndice \ref{graficas} se incluyen gráficas que muestran los resultados
obtenidos para cada \emph{dataset} de forma detallada. Lo importante, sin embargo, es
\textbf{observar los resultados obtenidos de manera general para cada muestra}, compuestas por 200 ejecucions del programa.

Se presentan las tablas \ref{tabla:count_1024} y \ref{tabla:resumen_1024} a modo de resumen para cada \emph{dataset}.
\clearpage

\begin{table}[h!]
    \centering
    \begin{tabular}{l r r r S S S}
    \strong{Dataset} & \strong{n} & \strong{N} & \strong{Est. media} &
    \strong{SE} & \textbf{T. medio ($ms$)} & \textbf{T. elem. ($\mu s$)}\\ \hline
    \newcounter{dataset}
\forloop{dataset}{1}{\value{dataset} < 10}{
\textbf{D\arabic{dataset}} &
\input{../data/D\arabic{dataset}/summary_1024.tex}
}
\end{tabular}
    \caption{Memoria fijada a 1024 bytes. Resumen de los resultados.}
    \label{tabla:resumen_1024}
\end{table}

\begin{table}[h!]
    \centering
    \begin{tabular}{l r r r r r r}
    \strong{Dataset} & \strong{[0, 1)} & \strong{[1, 5)} & \strong{[5, 10)} &
    \strong{[10, 15)} & \textbf{[15, 20)} & \textbf{[20, 100]} \\ \hline
\forloop{dataset}{1}{\value{dataset} < 10}{
\textbf{D\arabic{dataset}} &
\input{../data/D\arabic{dataset}/count_1024.tex}
}
\end{tabular}
    \caption{Memoria fijada a 1024 bytes. Clasificación de las ejecuciones según el error relativo (\%).}
    \label{tabla:count_1024}
\end{table}

Si se toma el \textbf{SE} de la tabla \ref{tabla:resumen_1024}, se puede observar que oscila alrededor del $3\%$, con picos de hasta el $3.6\%$ y un mínimo en $2.7\%$.

Estos valores no parecen seguir ningún tipo de relación con \textbf{N}, más allá del hecho que el \emph{dataset} con la menor \textbf{N}, \textit{D1}, es el que tiene un \textbf{SE} menor. No obstante, el mayor \emph{dataset}, \textit{D8}, tiene un valor por debajo de la media, por lo que \textbf{N} no parece influir en el \textbf{SE}.
\\

Por su parte, los \textbf{tiempos medios} sí que tienen un relación, bastante obvia, con \textbf{N}: cuanto mayor sea \textbf{N} más tardará el programa. Siendo una relación tan lógica no se profundizará más.
\\

Los \textbf{tiempos por elemento}, al contrario, no siguen ningún tipo de relación aparente con \textbf{N}, por lo que deben estar influenciados por otros factores no tan evidentes.
\\

Finalmente, si se observa la tabla \ref{tabla:count_1024}, se observa que los valores de \textbf{SE} están concentrados, en todos los \emph{datasets}, en el intervalo $[0,1)$ y $[1,5)$, con tan solo 2 valores en total por encima del $5\%$. De esto se extrae que el programa no sólo tiene un \textbf{SE} medio próximo al $3\%$, si no que sus valores individuales también le son próximos. Este dato es muy positivo, ya que permite comprobar que el programa no precisa de muchas ejecuciones para dar una estimación precisa, ya que cada ejecución tiene un \textbf{SE} razonablemente bajo.

\subsection{Influencia de la memoria disponible}

\begin{table}[h!]
    \centering
    \begin{tabular}{l r r r S S S}
    \strong{Memoria (bytes)} & \strong{n} & \strong{N} & \strong{Est. media} &
    \strong{SE} & \textbf{T. medio ($ms$)} & \textbf{T. elem. ($\mu s$)}\\ \hline

\textbf{32} & \input{../data/D1/summary_32.tex}
\textbf{64} & 3185 & 17219 & 3182 & 0.129 & 4.392 & 0.255\\ \hline

\textbf{128} & 3185 & 17219 & 3181 & 0.085 & 4.631 & 0.269\\ \hline

\textbf{256} & \input{../data/D1/summary_256.tex}
\textbf{512} & 3185 & 17219 & 3187 & 0.049 & 4.319 & 0.251\\ \hline

\textbf{1024} & 3185 & 17219 & 3213 & 0.035 & 3.915 & 0.227\\ \hline

\textbf{2048} & \input{../data/D1/summary_2048.tex}
\textbf{4096} & 3185 & 17219 & 3182 & 0.015 & 3.795 & 0.220\\ \hline

\textbf{8192} & 3185 & 17219 & 3186 & 0.011 & 4.388 & 0.255\\ \hline

\textbf{16384} & 3185 & 17219 & 3186 & 0.007 & 4.287 & 0.249\\ \hline


\end{tabular}
    \caption{Influencia de la memoria sobre el dataset D1. Resumen de resultados.}
    \label{tabla:resumen_1024}
\end{table}

\begin{table}[h!]
    \centering
    \begin{tabular}{l r r r r r r}
    \strong{Memoria} & \strong{[0, 1)} & \strong{[1, 5)} & \strong{[5, 10)} &
    \strong{[10, 15)} & \textbf{[15, 20)} & \textbf{[20, 100]} \\ \hline

\textbf{32} & 9 & 27 & 43 & 40 & 25 & 56\\ \hline

\textbf{64} & 9 & 40 & 59 & 46 & 28 & 18\\ \hline

\textbf{128} & 13 & 81 & 60 & 31 & 10 & 5\\ \hline

\textbf{256} & 23 & 74 & 70 & 28 & 2 & 3\\ \hline

\textbf{512} & 26 & 126 & 45 & 3 & 0 & 0\\ \hline

\textbf{1024} & 62 & 122 & 16 & 0 & 0 & 0\\ \hline

\textbf{2048} & 70 & 127 & 3 & 0 & 0 & 0\\ \hline

\textbf{4096} & 106 & 94 & 0 & 0 & 0 & 0\\ \hline

\textbf{8192} & 146 & 54 & 0 & 0 & 0 & 0\\ \hline

\textbf{16384} & \input{../data/D1/count_16384.tex}

\end{tabular}
    \caption{Influencia de la memoria sobre el dataset D1. Clasificación de las ejecuciones según el error relativo (\%).}
    \label{tabla:resumen_1024}
\end{table}
\\

Para estudiar el impacto de la \textbf{memoria disponible} en los resultados se ha usado el \emph{dataset} \textit{D1}, con valores de memoria que oscilan entre los $32 bytes$ y los $16 Kb$. En la tabla \ref{tabla:resumen_1024} se observa que en todos los casos la estimación es razonablemente precisa, siendo incluso tan precisa la estimación con $64 bytes$ como la que disponía de $4 Kb$. Esto podria llevar a la conclusión, errónea, de que el tamaño de la memoria es irrelevante para el resultado final. No obstante, si se comprueba el \textbf{SE} queda claro que cuanto mayor sea la memoria usada menor \textbf{SE} hay. Si se observa la tabla 4, en la que se muestran los intervalos en los que se encuentran los \textbf{SE} de cada muestra, se puede apreciar que los \textbf{SE} de las ejecuciones con memorias más reducidas se concentran en los valores superiores al $10\%$ de error. No es hasta que la \textbf{memoria disponible} aumenta hasta los 512 bytes que el \textbf{SE} se concentra por debajo del $10\%$, y solo a partir de los $4 Kb$ dejan de haber \textbf{SE} por encima del $5\%$. Por tanto, para obtener los mismos resultados que usando más memoria se deberían usar muchas más muestras, cosa que enlantecería el proceso, tal vez más de lo posible.

En general se podría concluir que el mínimo de \textbf{memoria necesaria} es de unos $512 bytes$, ya que su \textbf{SE} es del $5\%$ aproximadamente, un valor aceptable en la mayoría de casos. Por otro lado, dependiendo de la importancia de la precisión, se puede aumentar o disminuir este valor.
% !TEX encoding = UTF-8 Unicode
% !TEX root = ../report.tex
% 
\section{Conclusiones}
\label{conclusiones}

Este documento es el resultado de una investigación realizada sobre un algoritmo de \textbf{estimación de la cardinalidad}:
\texttt{HyperLogLog}. En él se ha descrito de manera general cuáles son las bases de este algoritmo y cómo funciona. Además, se
ha puesto a prueba su precisión bajo un análisis estadístico.

Las \textbf{conclusiones} obtenidas a partir de este ánalisis son las  siguientes:

\begin{itemize}
	\item Las estimaciones realizadas son igualmente \textbf{precisas} independientemente del \emph{dataset}.
	\item \textbf{A más memoria disponible, más precisión}. Sin embargo, mejorar la precisión necesita cada vez de más memoria, hasta un punto en que la mejora puede considerarse inapreciable.
	\item Para obtener estimaciones razonablemente \textbf{precisas}, por debajo del 10\% de error relativo, y en una sola ejecución se
necesitan por lo menos 512 bytes de \textbf{memoria}, pero los mejores resultados se empiezan a obtener a partir de
\textbf{1024 bytes} de \textbf{memoria disponible}.
	\item Con \textbf{memorias reducidas} es posible obtener \textbf{precisiones muy similares} a las obtenidas con
\textbf{memorias mayores} si se lleva a cabo una media entre un \textbf{número suficiente de ejecuciones}, con la desventaja del
incremento en \textbf{tiempo total de ejecución}.
	\item El programa es \textbf{rápido}, siendo capaz de procesar un \emph{dataset} de \textbf{más de 7 millones de elementos en medio segundo}.
\end{itemize}

Finalmente, se concluye que \texttt{HyperLogLog} funciona de manera \textbf{eficaz} y \textbf{eficiente}, obteniendo unos resultados
considerablemente \textbf{precisos} sin necesitar de mucho \textbf{tiempo} ni \textbf{memoria}.


\nocite{*}
\printbibliography[heading=bibintoc,heading=bibnumbered]

\end{document}
