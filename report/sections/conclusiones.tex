% !TEX encoding = UTF-8 Unicode
% !TEX root = ../report.tex
% 
\section{Conclusiones}
\label{conclusiones}

Una vez estudiadas las gráficas y tablas, hemos llegado a las siguientes \textbf{conclusiones}:


\begin{itemize}
	\item Las estimaciones realizadas pueden ser igualmente \textbf{precisas} con \emph{datasets} grandes o pequeños.
	
	\item Para obtener estimaciones razonablemente \textbf{precisas ($SE\leq5\%$)} con una sola ejecución se necesitan por lo menos $512 bytes$ de \textbf{memoria}, pero los mejores resultados se empiezan a obtener a partir de $1 Kb$ de \textbf{memoria disponible}.
	
	\item Con \textbf{memorias reducidas} es posible obtener \textbf{precisiones similares o incluso superiores} a las obtenidas con \textbf{memorias mayores} si se lleva a cabo una media entre un \textbf{número suficiente de ejecuciones}, con la desventaja del incremento en \textbf{tiempo de ejecución}.

	\item El programa es razonablemente \textbf{rápido}, siendo capaz de procesar un \emph{dataset} de más de $7\times10^6$ elementos en medio segundo.
\end{itemize}

Finalmente, podemos concluir que nuestro programa lleva a cabo su cometido de manera \textbf{eficaz}, ya que obtiene unos resultados considerablemente \textbf{precisos}, y \textbf{eficiente}, ya que no precisa de mucho \textbf{tiempo} ni \textbf{memoria}.