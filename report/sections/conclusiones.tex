% !TEX encoding = UTF-8 Unicode
% !TEX root = ../report.tex
% 
\section{Conclusiones}
\label{conclusiones}

Este documento es el resultado de una investigación realizada sobre un algoritmo de \textbf{estimación de la cardinalidad}:
\texttt{HyperLogLog}. En él se ha descrito de manera general cuáles son las bases de este algoritmo y cómo funciona. Además, se
ha puesto a prueba su precisión bajo un análisis estadístico.

Las \textbf{conclusiones} obtenidas a partir de este ánalisis son las  siguientes:

\begin{itemize}
	\item Las estimaciones realizadas son igualmente \textbf{precisas} independientemente del \emph{dataset}.
	\item \textbf{A más memoria disponible, más precisión}. Sin embargo, mejorar la precisión necesita cada vez de más memoria, hasta un punto en que la mejora puede considerarse inapreciable.
	\item Para obtener estimaciones razonablemente \textbf{precisas}, por debajo del 10\% de error relativo, y en una sola ejecución se
necesitan por lo menos 512 bytes de \textbf{memoria}, pero los mejores resultados se empiezan a obtener a partir de
\textbf{1024 bytes} de \textbf{memoria disponible}.
	\item Con \textbf{memorias reducidas} es posible obtener \textbf{precisiones muy similares} a las obtenidas con
\textbf{memorias mayores} si se lleva a cabo una media entre un \textbf{número suficiente de ejecuciones}, con la desventaja del
incremento en \textbf{tiempo total de ejecución}.
	\item El programa es \textbf{rápido}, siendo capaz de procesar un \emph{dataset} de \textbf{más de 7 millones de elementos en medio segundo}.
\end{itemize}

Finalmente, se concluye que \texttt{HyperLogLog} funciona de manera \textbf{eficaz} y \textbf{eficiente}, obteniendo unos resultados
considerablemente \textbf{precisos} sin necesitar de mucho \textbf{tiempo} ni \textbf{memoria}.
