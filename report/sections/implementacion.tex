% !TEX encoding = UTF-8 Unicode
% !TEX root = ../report.tex
% 
\section{Implementación}

\subsection{Función de hash}
\label{implementacion:hash}

Para implementar el algoritmo se ha usado una función de hash llamada \texttt{djb2} ~\cite{hash:djb2}.
Se sabe que esta función de hash
es \textbf{muy buena} cuando el \textbf{dominio} es el conjunto de todas las \textbf{cadenas de carácteres ASCII},
hecho que ocurre precisamente en los \textbf{juegos de pruebas disponibles}.

\begin{algorithm}[h]
\caption{Función de hash djb2}
\textbf{let} $a \in \{0, 1\}^{64}$ chosen randomly

\textbf{input} $str \in \Sigma^*$

\textbf{output} $hash \in \{0, 1\}^{64}$
\begin{algorithmic}
    \STATE $hash = 5381$
    \STATE $i := 0$
    \WHILE{$i < length(str)$}
        \STATE $hash := hash * 33 + str[i]$
        \STATE $i := i + 1$
    \ENDWHILE
    \RETURN $a \cdot hash$
\end{algorithmic}
\end{algorithm}

El \textbf{coste temporal} de aplicar esta función de hash a una cadena de cáracteres $str$ es $\bigO{length(str)}$.

La constante $a$ es un número de 64 bits \textbf{escogido aleatoriamente para cada ejecución del programa}.
Esto es necesario para poder realizar un \textbf{estudio estadístico del estimador de cardinalidad} y, además, se evita la posibilidad
de tener juegos de pruebas generados intencionadamente para hacer fallar la estimación, es decir, \textbf{no hay casos peores}.

La función \texttt{djb2} sigue unos principios parecidos a los de un \texttt{Linear Congruental Generator}  (\emph{LCG} para abreviar),
que tiene la forma:
$$X_{n+1} \equiv \left( h \cdot X_n + c \right)~~\pmod{m}$$

Usando $h=33$ y $m=2^{64}$. El valor elegido para $h$ es relativamente arbitrario, pero debe cumplir:

\begin{itemize}
	\item $\textbf{h-1}$ \textbf{es divisible por todos los factores primos de} $\textbf{m}$.
En el caso de $h=33$, $h-1$ es $32$, divisible por el único factor primo de $2^{64}$, el $2$.

	\item $\textbf{h-1}$ \textbf{es múltiplo de 4 si} $\textbf{m}$ \textbf{también lo es}.
En el caso de $a=33$, como $m$ es múltiplo de 4, $h-1$ también lo debe ser, y efectivamente: $32 \! \mod 4 = 0$.
\end{itemize}

Un \emph{LCG} es un algoritmo que produce una secuencia de números \emph{pseudo-aleatorios} calculados mediante una
\textbf{ecuación lineal}. Una de sus principales ventajas es que necesita un \textbf{número muy reducido de bits para mantener el
estado}, por lo que es una buena elección en el entorno en el que se utiliza, donde se dispone de \textbf{poca memoria}.

Puede consultarse el \textbf{código \texttt{C++}} implementando esta función en el \textbf{apéndice \ref{codigo:hash}}.

\subsection{Algoritmo \texttt{HyperLogLog}}
\label{implementacion:algoritmo}

Dado que se ha decidido utilizar una \textbf{función de hash de 64} bits para \textbf{minimizar el número de colisiones},
resulta innecesario aplicar la
\textbf{corrección para estimaciones altas} vista en la sección: \nameref{investigacion:descripcion} (\ref{investigacion:descripcion}).
Esto es así
porque, en la práctica, observar estimaciones cercanas a $2^{64}$ es poco común y, en esos casos, quizás sería mejor utilizar más bits
en la función de hash antes que aplicar la corrección.

El algoritmo \ref{algoritmo:hyper2} muestra en pseudocódigo la \textbf{versión final} que se ha implementado en \texttt{C++},
\textbf{disponible en el apéndice \ref{codigo:hyperloglog}}, y sobre la que se ha realizado el \textbf{análisis estadístico}
de la sección \ref{analisis}.

\begin{algorithm}[h!]
\caption{\texttt{HyperLogLog} para funciones de hash de 64 bits}
\label{algoritmo:hyper2}
\textit{Let $h: D\rightarrow \{0,1\}^{64}$ hash data from D to binary 32-bit word.}

\textit{Let $\rho(s)$ be the position of the leftmost 1-bit of s: e.g.,
$\rho(1...) = 1, \rho(0001...) = 4, \rho(0^K) = K + 1$.}

\textbf{define} $\alpha_{16}=0.673;\alpha_{32}=0.697;\alpha_{64}=0.709;\alpha_m=0.7213/(1+1.079/m)$
for $m \geq 128;$

\textbf{Program \texttt{HYPERLOGLOG}} (\textbf{input} $M$: multiset of items from domain $D$).

\textbf{assume} $m=2^b$ with $ b\in[4..16]$.

\textbf{initialize} a collection of $m$ registers, $M[1],...,M[m]$, to 0;

\begin{algorithmic}
    \FOR{$v\in M$}
            \STATE $x  := h(v)$
            \STATE $j   := 1 + (x_1 x_2 ... x_b)_2$ \COMMENT{binary address determined by the first b bits of x}
            \STATE $w := x_{b+1} x_{b+2} ... $
            \STATE $M[j] := max(M[j],\rho(w))$
    \ENDFOR

    \STATE $E:=\alpha _m m^2·\left(\sum\limits_{j=1}^m 2^{-M[j]}\right)^{-1}$ \COMMENT{the raw HyperLogLog estimate}
    \IF{$E \leq \frac{5}{2}m$}
        \STATE $V :=$ the number of registers equal to $0$
        \STATE \algorithmicif\ $V \neq 0$ \algorithmicthen\ $E* := m \cdot log(m / V)$ \algorithmicelse\ $E* := E$
        \COMMENT{small range correction}
    \ELSE
        \STATE $E*:=E$
    \ENDIF
    \RETURN{cardinality estimate E* with typical relative error $\pm$ 1.04/$\sqrt{m}$}
\end{algorithmic}
\end{algorithm}

\newpage