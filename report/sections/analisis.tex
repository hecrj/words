% !TEX encoding = UTF-8 Unicode
% !TEX root = ../report.tex
% 
\section{Análisis estadístico del algoritmo}

\subsection{Versión implementada}

\subsubsection{Función de hash}
Para implementar el algoritmo se ha usado una función de hash llamada \texttt{djb2} ~\cite{hash:djb2}.
Se sabe que esta función de hash
es \textbf{muy buena} cuando el \textbf{dominio} es el conjunto de todas las \textbf{cadenas de carácteres ASCII},
hecho que ocurre precisamente en los \textbf{juegos de pruebas disponibles}.

\begin{algorithm}[h]
\caption{Función de hash djb2}
\textbf{let} $a \in \{0, 1\}^{64}$ chosen randomly

\textbf{input} $str \in \Sigma^*$

\textbf{output} $hash \in \{0, 1\}^{64}$
\begin{algorithmic}
    \STATE $hash = 5381$
    \STATE $i := 0$
    \WHILE{$i < length(str)$}
        \STATE $hash := hash * 33 + str[i]$
        \STATE $i := i + 1$
    \ENDWHILE
    \RETURN $a \cdot hash$
\end{algorithmic}
\end{algorithm}

La constante $a$ es un número 64 bits \textbf{escogido aleatoriamente para cada ejecución del programa}.
Esto es necesario para poder realizar un \textbf{estudio estadístico del estimador de cardinalidad} y, además, se evita la posibilidad
de tener juegos de pruebas generados intencionadamente para hacer fallar la estimación, es decir, \textbf{no hay casos peores}.

\texttt{djb2} sigue unos principios parecidos a los de un \texttt{Linear Congruental generator}  (\emph{LCG} para abreviar),
que tiene la forma:
$$X_{n+1} \equiv \left( h \cdot X_n + c \right)~~\pmod{m}$$

Donde $h=33$ y $m=2^{64}$. El valor elegido para $h$ es relativamente arbitrario, pero debe cumplir:

\begin{itemize}
	\item $\textbf{h-1}$ \textbf{es divisible por todos los factores primos de} $\textbf{m}$.
En el caso de $h=33$, $h-1$ es $32$, divisible por el único factor primo de $2^{64}$, el $2$.

	\item $\textbf{h-1}$ \textbf{es múltiplo de 4 si} $\textbf{m}$ \textbf{también lo es}.
En el caso de $a=33$, como $m$ es múltiplo de 4, $h-1$ también lo debe ser, y efectivamente: $32 \! \mod 4 = 0$.
\end{itemize}

Un \emph{LCG} es un algoritmo que produce una secuencia de números \emph{pseudo-aleatorios} calculados mediante una
\textbf{ecuación lineal}. Una de sus principales ventajas es que necesita un \textbf{número muy reducido de bits para mantener el
estado}, por lo que es una buena elección en el entorno en el que se utiliza, donde se dispone de \textbf{poca memoria}.

