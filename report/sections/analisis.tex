% !TEX encoding = UTF-8 Unicode
% !TEX root = ../report.tex
% 
\section{Análisis estadístico del algoritmo}
\label{analisis}

\subsection{Obtención de las muestras}
Para la \textbf{obtención de las muestras} necesarias para el \textbf{análisis estadístico} del algoritmo \texttt{HyperLogLog}
se ha utilizando un script programado en \textbf{Ruby}. Puede verse el código en el apéndice ...

El script recibe como parámetros:
\begin{enumerate}
\item Los datasets de los que se quieren recoger muestras (por defecto \textbf{todos})
\item Los casos por muestra (por defecto \textbf{200})
\item Las memorias de las que recoger datas (por defecto \textbf{1024})
\item El directorio donde guardar resultados (por defecto \textbf{data})
\end{enumerate}

El script \textbf{compila y ejecuta el código en \texttt{C++} que implementa el algoritmo \texttt{HyperLogLog} para recoger los datos
y crea archivos con los resultados de cada muestra}, en forma de tabla. Finalmente, pregunta si se desean
\textbf{generar gráficas} a partir de los resultados obtenidos y, si se responde positivamente, ejecuta un \textbf{script en R que las genera autómaticamente} (disponible en el apéndice ...).

Para el primer análisis: \textbf{\nameref{analisis:mem_1024}}, las muestras se han obtenido utilizando los parámetros por defecto del script.

Para el segundo análisis: \textbf{\nameref{analisis:D1_mem}}, las muestras se han obtenido utilizando los parámetros:
\begin{description}
\item[Dataset] D1
\item[Memorias] 32, 64, 128, 256, 512, 1024, 2048, 4096, 8192, 16384
\end{description}

Y los otros parámetros con su valor por defecto.

\subsection{Memoria fijada a 1024 bytes}
\label{analisis:mem_1024}

En esta sección se estudia el comportamiento del algoritmo \texttt{HyperLogLog} en \textbf{9 \emph{datasets}} distintos limitando
la cantidad de memoria utilizada a \textbf{1024 bytes}. En el apéndice \ref{graficas} se incluyen gráficas que muestran los resultados
obtenidos para cada \emph{dataset} de forma detallada. Lo importante, sin embargo, es
\textbf{observar los resultados obtenidos de manera general para cada muestra}.

Se presentan las tablas \ref{tabla:count_1024} y \ref{tabla:resumen_1024} a modo de resumen para cada \emph{dataset}.

\begin{table}[h!]
    \centering
    \begin{tabular}{l r r r S S S}
    \strong{Dataset} & \strong{n} & \strong{N} & \strong{Est. media} &
    \strong{SE} & \textbf{T. medio ($ms$)} & \textbf{T. elem. ($\mu s$)}\\ \hline
    \newcounter{dataset}
\forloop{dataset}{1}{\value{dataset} < 10}{
\textbf{D\arabic{dataset}} &
\input{../data/D\arabic{dataset}/summary_1024.tex}
}
\end{tabular}
    \caption{Memoria fijada a 1024 bytes. Resumen de los resultados.}
    \label{tabla:resumen_1024}
\end{table}

\begin{table}[h!]
    \centering
    \begin{tabular}{l r r r r r r}
    \strong{Dataset} & \strong{[0, 1)} & \strong{[1, 5)} & \strong{[5, 10)} &
    \strong{[10, 15)} & \textbf{[15, 20)} & \textbf{[20, 100]} \\ \hline
\forloop{dataset}{1}{\value{dataset} < 10}{
\textbf{D\arabic{dataset}} &
\input{../data/D\arabic{dataset}/count_1024.tex}
}
\end{tabular}
    \caption{Memoria fijada a 1024 bytes. Clasificación de las ejecuciones según el error relativo (\%).}
    \label{tabla:count_1024}
\end{table}

Si se toma el \textbf{SE} de la tabla \ref{tabla:resumen_1024}, se puede observar que \textbf{no parece verse afectado por el \emph{dataset} utilizado}, puesto que en todos ellos oscila alrededor del $3\%$.

Por su parte, los \textbf{tiempos medios} sí que tienen un relación bastante obvia con \textbf{N}
\textbf{cuanto mayor sea \textbf{N} más tardará el programa}. Esto es de esperar, puesto que el coste de un algoritmo suele
depender directamente de la entrada que recibe.

\textbf{Los tiempos por elemento varían considerablemente dependiendo del dataset}. Esto también es de esperar, puesto que
un dataset puede contener, en media, más elementos de \textbf{una longitud mayor} respecto a otro y, por lo tanto,
\textbf{aplicar la función de hash djb2 (sección \ref{implementacion:hash}) tendrá un coste mayor}.

Finalmente, en la tabla \ref{tabla:count_1024} se observa que los valores de \textbf{los errores relativos} están
concentrados, en todos los \emph{datasets}, en el intervalo $[0,5)$, con muy pocos valores en total por encima del $5\%$.
Este dato es muy positivo porque demuestra que el programa no precisa de muchas ejecuciones para dar una
estimación precisa, ya que cada ejecución tiene un \textbf{error relativo} razonablemente bajo. Por otro lado, este dato también
nos muestra que \textbf{la precisión del algoritmo para una ejecución en concreto no depende del \emph{dataset}}.

\subsection{Influencia de la memoria disponible}
\label{analisis:D1_mem}

Para estudiar el impacto de la \textbf{memoria disponible} en los resultados se ha usado el \textbf{\emph{dataset} D1},
con valores de memoria $m \in \{ 2^x \: | \; 5 \leq x \leq 14 \}$.

Se presentan las figuras \ref{figura:mem_estimation}, \ref{figura:mem_errors} y \ref{figura:mem_time} donde se puede observar la
\textbf{influencia de la memoria} sobre la \textbf{estimación media}, el \textbf{error estándar} y el \textbf{tiempo medio},
respectivamente.

\begin{figure}[h!]
    \centering
        \includegraphics[width=0.64\textwidth]{../figs/D1/mem_estimation_rel.pdf}
        \caption{Influencia de la memoria disponible sobre la estimación media}
    \label{figura:mem_estimation}
\end{figure}

\begin{figure}[h!]
    \centering
        \includegraphics[width=0.64\textwidth]{../figs/D1/mem_errors_rel.pdf}
        \caption{Influencia de la memoria disponible sobre el error estándar}
    \label{figura:mem_errors}
\end{figure}

\begin{figure}[h!]
    \centering
        \includegraphics[width=0.64\textwidth]{../figs/D1/mem_time_rel.pdf}
        \caption{Influencia de la memoria disponible sobre el tiempo medio}
    \label{figura:mem_time}
\end{figure}

\clearpage

En la figura \ref{figura:mem_estimation} se observa que en todos los casos la estimación media es muy precisa. Esto podria llevar a
la conclusión \textbf{errónea} de que el tamaño de la memoria es irrelevante para el resultado final. No obstante, en la figura
\ref{figura:mem_errors} se observa que \textbf{cuanto mayor es la memoria usada menor es el \textbf{SE}}.
De hecho, puede observarse
que el \textbf{SE} disminuye considerablemente conforme la memoria aumenta al principio pero, sin embargo, cada vez disminuye
menos. Más exactamente, puede apreciarse como \textbf{los distintos valores obtenidos obedecen perfectamente el error estándar
esperado para el estimador de \texttt{HyperLogLog}}:
$$SE = \frac{1.04}{\sqrt{m}}$$

\begin{table}[h!]
    \centering
    \begin{tabular}{l r r r S S S}
    \strong{Memoria (bytes)} & \strong{n} & \strong{N} & \strong{Est. media} &
    \strong{SE} & \textbf{T. medio ($ms$)} & \textbf{T. elem. ($\mu s$)}\\ \hline

\textbf{32} & \input{../data/D1/summary_32.tex}
\textbf{64} & 3185 & 17219 & 3182 & 0.129 & 4.392 & 0.255\\ \hline

\textbf{128} & 3185 & 17219 & 3181 & 0.085 & 4.631 & 0.269\\ \hline

\textbf{256} & \input{../data/D1/summary_256.tex}
\textbf{512} & 3185 & 17219 & 3187 & 0.049 & 4.319 & 0.251\\ \hline

\textbf{1024} & 3185 & 17219 & 3213 & 0.035 & 3.915 & 0.227\\ \hline

\textbf{2048} & \input{../data/D1/summary_2048.tex}
\textbf{4096} & 3185 & 17219 & 3182 & 0.015 & 3.795 & 0.220\\ \hline

\textbf{8192} & 3185 & 17219 & 3186 & 0.011 & 4.388 & 0.255\\ \hline

\textbf{16384} & 3185 & 17219 & 3186 & 0.007 & 4.287 & 0.249\\ \hline


\end{tabular}
    \caption{Influencia de la memoria sobre el dataset D1. Resumen de resultados.}
    \label{tabla:resumen_1024}
\end{table}

\begin{table}[h!]
    \centering
    \begin{tabular}{l r r r r r r}
    \strong{Memoria} & \strong{[0, 1)} & \strong{[1, 5)} & \strong{[5, 10)} &
    \strong{[10, 15)} & \textbf{[15, 20)} & \textbf{[20, 100]} \\ \hline

\textbf{32} & 9 & 27 & 43 & 40 & 25 & 56\\ \hline

\textbf{64} & 9 & 40 & 59 & 46 & 28 & 18\\ \hline

\textbf{128} & 13 & 81 & 60 & 31 & 10 & 5\\ \hline

\textbf{256} & 23 & 74 & 70 & 28 & 2 & 3\\ \hline

\textbf{512} & 26 & 126 & 45 & 3 & 0 & 0\\ \hline

\textbf{1024} & 62 & 122 & 16 & 0 & 0 & 0\\ \hline

\textbf{2048} & 70 & 127 & 3 & 0 & 0 & 0\\ \hline

\textbf{4096} & 106 & 94 & 0 & 0 & 0 & 0\\ \hline

\textbf{8192} & 146 & 54 & 0 & 0 & 0 & 0\\ \hline

\textbf{16384} & \input{../data/D1/count_16384.tex}

\end{tabular}
    \caption{Influencia de la memoria sobre el dataset D1. Clasificación de las ejecuciones según el error relativo (\%).}
    \label{tabla:count_1024}
\end{table}

Si se analiza la tabla \ref{tabla:count_1024} se puede apreciar que los
\textbf{errores relativos} de las ejecuciones con memorias más reducidas se concentran en los valores superiores al $10\%$. No es
hasta que la \textbf{memoria disponible} aumenta hasta los 512 bytes que el \textbf{error relativo} se concentra por debajo del
$10\%$, y solo a partir de los 4 Kbytes dejan de aparecer ejecuciones por encima del $5\%$. Por tanto, para obtener una estimación
precisa con poca memoria sería necesario realizar diversas ejecuciones, aumentando así el tiempo total, tal vez más de lo
posible.

En general se podría concluir que el mínimo de \textbf{memoria necesaria} es de unos 512 bytes, ya que su \textbf{SE} es del $5\%$
aproximadamente, un valor aceptable en la mayoría de casos. Por otro lado, dependiendo de la importancia de la precisión, se puede
aumentar o disminuir este valor.
