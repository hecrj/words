% !TEX encoding = UTF-8 Unicode
% !TEX root = ../report.tex
% 
\section{Análisis estadístico del algoritmo}
\label{analisis}

\subsection{Memoria fijada a 1024 bytes}

En esta sección se estudia el comportamiento del algoritmo \texttt{HyperLogLog} en \textbf{9 \emph{datasets}} distintos limitando
la cantidad de memoria utilizada a \textbf{1024 bytes}. En el apéndice \ref{graficas} se incluyen gráficas que muestran los resultados
obtenidos para cada \emph{dataset} de forma detallada. Lo importante, sin embargo, es
\textbf{observar los resultados obtenidos de manera general para cada muestra}.

Se presentan las tablas \ref{tabla:resumen_1024} y \ref{tabla:count_1024} a modo de resumen para cada \emph{dataset}.
\clearpage

\begin{table}[h!]
    \centering
    \begin{tabular}{l r r r S S S}
    \strong{Dataset} & \strong{n} & \strong{N} & \strong{Est. media} &
    \strong{SE} & \textbf{T. medio ($ms$)} & \textbf{T. elem. ($\mu s$)}\\ \hline
    \newcounter{dataset}
\forloop{dataset}{1}{\value{dataset} < 10}{
\textbf{D\arabic{dataset}} &
\input{../data/D\arabic{dataset}/summary_1024.tex}
}
\end{tabular}
    \caption{Memoria fijada a 1024 bytes. Resumen de los resultados.}
    \label{tabla:resumen_1024}
\end{table}

\begin{table}[h!]
    \centering
    \begin{tabular}{l r r r r r r}
    \strong{Dataset} & \strong{[0, 1)} & \strong{[1, 5)} & \strong{[5, 10)} &
    \strong{[10, 15)} & \textbf{[15, 20)} & \textbf{[20, 100]} \\ \hline
\forloop{dataset}{1}{\value{dataset} < 10}{
\textbf{D\arabic{dataset}} &
\input{../data/D\arabic{dataset}/count_1024.tex}
}
\end{tabular}
    \caption{Memoria fijada a 1024 bytes. Clasificación de las ejecuciones según el error relativo (\%).}
    \label{tabla:count_1024}
\end{table}

[Aquí comentario sobre el SE cerca del valor esperado y otras cosas interesantes que puedas observar]

\subsection{Influencia de la memoria disponible}

\begin{table}[h!]
    \centering
    \begin{tabular}{l r r r S S S}
    \strong{Memoria} & \strong{n} & \strong{N} & \strong{Est. media} &
    \strong{SE} & \textbf{T. medio ($ms$)} & \textbf{T. elem. ($\mu s$)}\\ \hline

\textbf{32} & 3185 & 17219 & 3209 & 0.194 & 4.849 & 0.282\\ \hline

\textbf{64} & 3185 & 17219 & 3182 & 0.129 & 4.392 & 0.255\\ \hline

\textbf{128} & 3185 & 17219 & 3181 & 0.085 & 4.631 & 0.269\\ \hline

\textbf{256} & 3185 & 17219 & 3190 & 0.077 & 4.341 & 0.252\\ \hline

\textbf{512} & 3185 & 17219 & 3187 & 0.049 & 4.319 & 0.251\\ \hline

\textbf{1024} & 3185 & 17219 & 3210 & 0.027 & 4.850 & 0.282\\ \hline

\textbf{2048} & 3185 & 17219 & 3186 & 0.026 & 4.901 & 0.285\\ \hline

\textbf{4096} & 3185 & 17219 & 3182 & 0.015 & 3.795 & 0.220\\ \hline

\textbf{8192} & 3185 & 17219 & 3186 & 0.024 & 4.420 & 0.257\\ \hline

\textbf{16384} & 3185 & 17219 & 3186 & 0.007 & 4.287 & 0.249\\ \hline


\end{tabular}
    \caption{Influencia de la memoria sobre el dataset D1. Resumen de resultados.}
    \label{tabla:resumen_1024}
\end{table}

\begin{table}[h!]
    \centering
    \begin{tabular}{l r r r r r r}
    \strong{Memoria} & \strong{[0, 1)} & \strong{[1, 5)} & \strong{[5, 10)} &
    \strong{[10, 15)} & \textbf{[15, 20)} & \textbf{[20, 100]} \\ \hline

\textbf{32} & 9 & 27 & 43 & 40 & 25 & 56\\ \hline

\textbf{64} & 9 & 40 & 59 & 46 & 28 & 18\\ \hline

\textbf{128} & 13 & 81 & 60 & 31 & 10 & 5\\ \hline

\textbf{256} & 23 & 74 & 70 & 28 & 2 & 3\\ \hline

\textbf{512} & 26 & 126 & 45 & 3 & 0 & 0\\ \hline

\textbf{1024} & 62 & 122 & 16 & 0 & 0 & 0\\ \hline

\textbf{2048} & 70 & 127 & 3 & 0 & 0 & 0\\ \hline

\textbf{4096} & 106 & 94 & 0 & 0 & 0 & 0\\ \hline

\textbf{8192} & 146 & 54 & 0 & 0 & 0 & 0\\ \hline

\textbf{16384} & 186 & 14 & 0 & 0 & 0 & 0\\ \hline


\end{tabular}
    \caption{Influencia de la memoria sobre el dataset D1. Clasificación de las ejecuciones según el error relativo (\%).}
    \label{tabla:resumen_1024}
\end{table}
